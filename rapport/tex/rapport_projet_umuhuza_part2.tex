% =============== CHAPITRE 3: ANALYSE ET CONCEPTION ===============
\chapter{Analyse et conception}

\section{Modélisation UML}

Cette section présente la modélisation complète du système Umuhuza à travers les 13 diagrammes UML standard. Tous les diagrammes ont été générés en utilisant PlantUML et sont disponibles dans le répertoire \texttt{/rapport/diagrams/}.

\subsection{Diagramme de classes}

Le diagramme de classes présente la structure complète du système avec ses 17 entités principales organisées en 5 packages Django.

\begin{figure}[H]
\centering
\includegraphics[width=\textwidth]{../diagrams/class_diagram.png}
\caption{Diagramme de classes complet du système Umuhuza}
\label{fig:class_diagram}
\end{figure}

\textbf{Packages principaux:}
\begin{itemize}
    \item \textbf{users:} Gestion des utilisateurs (User, VerificationCode, UserBadge, ActivityLog)
    \item \textbf{listings:} Gestion des annonces (Listing, Category, ListingImage, PricingPlan, UserSubscription, RatingReview, Favorite, ReportMisconduct)
    \item \textbf{messaging:} Système de messagerie (Chat, Message)
    \item \textbf{notifications:} Notifications (Notification)
    \item \textbf{payments:} Paiements et dealers (Payment, DealerApplication, DealerDocument)
\end{itemize}

\textbf{Relations clés:}
\begin{itemize}
    \item Un utilisateur peut créer plusieurs annonces (1-N)
    \item Une annonce appartient à une catégorie (N-1)
    \item Une annonce peut avoir plusieurs images (1-N)
    \item Un chat est lié à une annonce et deux utilisateurs (N-1-1)
    \item Un paiement est lié à un utilisateur et un plan tarifaire (N-1-1)
\end{itemize}

% NOTE: Remplacer par les images PNG générées depuis les fichiers .puml
% Pour générer: java -jar plantuml.jar class_diagram.puml

\subsection{Diagramme de composants}

\begin{figure}[H]
\centering
\includegraphics[width=0.9\textwidth]{../diagrams/architecture_components.png}
\caption{Architecture en composants du système}
\label{fig:components}
\end{figure}

Le système est organisé en trois couches principales:
\begin{enumerate}
    \item \textbf{Frontend Layer:} Application React SPA avec React Router, Context API pour la gestion d'état, et Axios comme client API.
    \item \textbf{Backend Layer:} Django 5.2 avec DRF, authentification JWT, CORS, et traitement d'images.
    \item \textbf{Data Layer:} PostgreSQL pour le stockage relationnel, S3/Local pour les fichiers.
\end{enumerate}

\subsection{Diagramme de déploiement}

\begin{figure}[H]
\centering
\includegraphics[width=\textwidth]{../diagrams/deployment_diagram.png}
\caption{Diagramme de déploiement de l'infrastructure}
\label{fig:deployment}
\end{figure}

\textbf{Infrastructure de déploiement:}
\begin{itemize}
    \item \textbf{Frontend:} Hébergé sur Vercel/Netlify avec CDN Cloudflare
    \item \textbf{Backend:} Serveurs d'application Django sur Railway/DigitalOcean avec Gunicorn
    \item \textbf{Base de données:} PostgreSQL managé (DigitalOcean Managed Database ou AWS RDS)
    \item \textbf{Stockage fichiers:} AWS S3 ou Cloudflare R2
    \item \textbf{Services externes:} SendGrid (email), Africa's Talking (SMS), Lumicash (paiement)
    \item \textbf{Monitoring:} Sentry (erreurs), Google Analytics (usage)
\end{itemize}

\subsection{Diagramme d'objets}

\begin{figure}[H]
\centering
% TODO: Générer diagramme d'objets montrant des instances concrètes
\fbox{\parbox{\textwidth}{
\textbf{Exemple d'instances concrètes:}\\
\\
\texttt{user1: User}\\
\texttt{- userid = 1}\\
\texttt{- email = "john@example.com"}\\
\texttt{- user\_firstname = "John"}\\
\texttt{- is\_seller = True}\\
\\
\texttt{listing1: Listing}\\
\texttt{- listing\_id = 123}\\
\texttt{- listing\_title = "Villa moderne à Bujumbura"}\\
\texttt{- listing\_price = 150000000}\\
\texttt{- userid = user1}\\
\\
\texttt{image1: ListingImage}\\
\texttt{- listimage\_id = 1}\\
\texttt{- listing\_id = listing1}\\
\texttt{- is\_primary = True}
}}
\caption{Exemple d'instances d'objets métier}
\label{fig:objects}
\end{figure}

\subsection{Diagramme de packages}

\begin{figure}[H]
\centering
% TODO: Générer diagramme de packages
\fbox{\parbox{0.8\textwidth}{
\textbf{Organisation des packages:}\\
\\
\texttt{backend/}\\
\texttt{├── umuhuza\_api/ (configuration Django)}\\
\texttt{├── users/ (package utilisateurs)}\\
\texttt{├── listings/ (package annonces)}\\
\texttt{├── messaging/ (package messagerie)}\\
\texttt{├── notifications/ (package notifications)}\\
\texttt{├── payments/ (package paiements)}\\
\texttt{└── media/ (fichiers uploadés)}
}}
\caption{Organisation en packages de l'application}
\label{fig:packages}
\end{figure}

\subsection{Diagramme de structure composite}

Le modèle User peut être vu comme une structure composite avec plusieurs sous-composants:
\begin{itemize}
    \item Profil de base (informations personnelles)
    \item Mécanisme d'authentification (mot de passe, tokens)
    \item Système de vérification (email, téléphone)
    \item Collection de badges
    \item Historique d'activité
\end{itemize}

\subsection{Diagramme de cas d'utilisation}

\begin{figure}[H]
\centering
\includegraphics[width=\textwidth]{../diagrams/use_case_diagram.png}
\caption{Diagramme de cas d'utilisation complet}
\label{fig:use_case}
\end{figure}

\textbf{Acteurs et leurs cas d'utilisation:}

\begin{table}[H]
\centering
\begin{tabular}{|l|p{10cm}|}
\hline
\textbf{Acteur} & \textbf{Cas d'utilisation} \\
\hline
Visiteur & Consulter annonces, Rechercher, Filtrer, S'inscrire, Se connecter \\
\hline
Acheteur & Contacter vendeur, Ajouter favoris, Envoyer message, Noter vendeur, Signaler \\
\hline
Vendeur & Créer annonce, Modifier annonce, Gérer photos, Consulter stats, Acheter plan premium \\
\hline
Dealer & Soumettre candidature, Créer annonces illimitées, Dashboard analytics, Gérer abonnement \\
\hline
Admin & Modérer annonces, Gérer utilisateurs, Traiter signalements, Approuver dealers, Consulter logs \\
\hline
\end{tabular}
\caption{Matrice acteurs/cas d'utilisation}
\end{table}

\subsection{Diagramme de séquence}

\subsubsection{Séquence d'inscription utilisateur}

\begin{figure}[H]
\centering
\includegraphics[width=\textwidth]{../diagrams/sequence_registration.png}
\caption{Diagramme de séquence - Inscription et vérification utilisateur}
\label{fig:seq_registration}
\end{figure}

\textbf{Étapes du processus:}
\begin{enumerate}
    \item L'utilisateur remplit le formulaire d'inscription
    \item Le frontend valide les données localement
    \item Une requête POST est envoyée à l'API avec les informations
    \item Le backend vérifie l'unicité de l'email
    \item Le mot de passe est hashé avec PBKDF2
    \item L'utilisateur est créé avec \texttt{is\_verified=False}
    \item Un code de vérification à 6 chiffres est généré
    \item Un email est envoyé avec le code
    \item L'utilisateur saisit le code reçu
    \item Le backend vérifie la validité du code
    \item Le statut \texttt{email\_verified} est mis à jour
    \item L'utilisateur est connecté automatiquement
\end{enumerate}

\subsubsection{Séquence de création d'annonce}

\begin{figure}[H]
\centering
\includegraphics[width=\textwidth]{../diagrams/sequence_create_listing.png}
\caption{Diagramme de séquence - Création d'annonce avec photos}
\label{fig:seq_listing}
\end{figure}

\subsubsection{Séquence de messagerie}

\begin{figure}[H]
\centering
\includegraphics[width=\textwidth]{../diagrams/sequence_messaging.png}
\caption{Diagramme de séquence - Messagerie entre acheteur et vendeur}
\label{fig:seq_messaging}
\end{figure}

\subsubsection{Séquence de paiement}

\begin{figure}[H]
\centering
\includegraphics[width=\textwidth]{../diagrams/sequence_payment.png}
\caption{Diagramme de séquence - Processus de paiement Lumicash}
\label{fig:seq_payment}
\end{figure}

\subsection{Diagramme de communication}

Les diagrammes de communication complètent les diagrammes de séquence en montrant la structure des interactions entre objets.

\textbf{Communication lors d'une création d'annonce:}
\begin{itemize}
    \item Frontend $\leftrightarrow$ Backend API: POST /api/listings/create/
    \item Backend API $\leftrightarrow$ Database: INSERT INTO LISTINGS
    \item Backend API $\leftrightarrow$ Storage: Upload images
    \item Backend API $\leftrightarrow$ Notification Service: Notify admin
\end{itemize}

\subsection{Diagramme d'activités}

\subsubsection{Activité: Inscription utilisateur}

\begin{figure}[H]
\centering
\includegraphics[width=0.7\textwidth]{../diagrams/activity_registration.png}
\caption{Diagramme d'activité - Processus d'inscription}
\label{fig:act_registration}
\end{figure}

\subsubsection{Activité: Création d'annonce}

\begin{figure}[H]
\centering
\includegraphics[width=0.7\textwidth]{../diagrams/activity_create_listing.png}
\caption{Diagramme d'activité - Création d'annonce avec vérification de quota}
\label{fig:act_listing}
\end{figure}

\subsubsection{Activité: Traitement de paiement}

\begin{figure}[H]
\centering
\includegraphics[width=0.7\textwidth]{../diagrams/activity_payment.png}
\caption{Diagramme d'activité - Processus de paiement}
\label{fig:act_payment}
\end{figure}

\subsection{Diagramme d'états (états-transitions)}

\begin{figure}[H]
\centering
\includegraphics[width=0.8\textwidth]{../diagrams/state_listing.png}
\caption{Diagramme d'états - Cycle de vie d'une annonce}
\label{fig:state_listing}
\end{figure}

\textbf{États possibles d'une annonce:}
\begin{itemize}
    \item \textbf{Draft:} Annonce en cours de création
    \item \textbf{Pending:} Soumise, en attente de modération admin
    \item \textbf{Active:} Approuvée et visible publiquement
    \item \textbf{Hidden:} Masquée temporairement par le vendeur
    \item \textbf{Sold:} Marquée comme vendue
    \item \textbf{Expired:} Date d'expiration dépassée
    \item \textbf{Rejected:} Rejetée par l'admin lors de la modération
    \item \textbf{Deleted:} Supprimée (soft delete)
\end{itemize}

\subsection{Diagramme de temporisation}

Le diagramme de temporisation illustre le comportement temporel du système de vérification:

\begin{itemize}
    \item \textbf{T0:} Génération du code de vérification
    \item \textbf{T0+15min:} Code expire si non utilisé
    \item \textbf{T0+3 tentatives:} Blocage temporaire du compte si 3 codes invalides
    \item \textbf{T0+24h:} Suppression automatique du compte si non vérifié
\end{itemize}

\subsection{Diagramme d'interaction global}

Le diagramme d'interaction global combine les différents diagrammes d'interaction pour montrer le flux complet d'une transaction:

\begin{enumerate}
    \item Inscription et vérification (diagramme de séquence)
    \item Création d'annonce (diagramme d'activité)
    \item Contact vendeur (diagramme de communication)
    \item Négociation par messages (diagramme de séquence)
    \item Paiement si plan premium (diagramme de séquence)
    \item Finalisation transaction (diagramme d'états)
\end{enumerate}

\section{Modélisation de la base de données}

\subsection{MCD (Modèle Conceptuel des Données)}

Le MCD présente les entités métier et leurs associations sans considérations techniques.

\begin{figure}[H]
\centering
\includegraphics[width=\textwidth]{../diagrams/er_diagram.png}
\caption{Modèle Conceptuel des Données (MCD)}
\label{fig:mcd}
\end{figure}

\textbf{Entités principales:}
\begin{itemize}
    \item USERS: Stocke les utilisateurs du système
    \item LISTINGS: Stocke les annonces
    \item CATEGORIES: Catégories d'annonces
    \item LISTING\_IMAGES: Photos des annonces
    \item CHATS: Conversations
    \item MESSAGES: Messages échangés
    \item PAYMENTS: Transactions financières
    \item DEALER\_APPLICATIONS: Candidatures dealer
    \item NOTIFICATIONS: Notifications utilisateur
\end{itemize}

\textbf{Cardinalités clés:}
\begin{itemize}
    \item USERS (1,N) -- CREATES -- (0,N) LISTINGS
    \item LISTINGS (1,1) -- BELONGS\_TO -- (0,N) CATEGORIES
    \item LISTINGS (1,1) -- HAS -- (1,10) LISTING\_IMAGES
    \item USERS (1,1) -- PARTICIPATES -- (0,N) CHATS
    \item CHATS (1,1) -- CONTAINS -- (0,N) MESSAGES
    \item USERS (1,1) -- MAKES -- (0,N) PAYMENTS
\end{itemize}

\subsection{MLD (Modèle Logique des Données)}

Le MLD traduit le MCD en schéma relationnel avec clés primaires et étrangères.

\textbf{Tables principales:}

\begin{lstlisting}[language=SQL, caption=Schéma relationnel (MLD)]
USERS(
    USERID PK,
    USER_EMAIL UNIQUE NOT NULL,
    USER_FIRSTNAME NOT NULL,
    USER_LASTNAME NOT NULL,
    PHONE_NUMBER NOT NULL,
    PASSWORD_HASH NOT NULL,
    USER_ROLE DEFAULT 'buyer',
    IS_SELLER DEFAULT FALSE,
    IS_DEALER DEFAULT FALSE,
    IS_VERIFIED DEFAULT FALSE,
    EMAIL_VERIFIED DEFAULT FALSE,
    DATE_JOINED TIMESTAMP
)

CATEGORIES(
    CAT_ID PK,
    CAT_NAME NOT NULL,
    SLUG UNIQUE NOT NULL,
    CAT_DESCRIPTION TEXT,
    IS_ACTIVE DEFAULT TRUE
)

LISTINGS(
    LISTING_ID PK,
    USERID FK -> USERS(USERID),
    CAT_ID FK -> CATEGORIES(CAT_ID),
    LISTING_TITLE NOT NULL,
    LIST_DESCRIPTION TEXT NOT NULL,
    LISTING_PRICE DECIMAL(15,2) NOT NULL,
    LIST_LOCATION VARCHAR(255) NOT NULL,
    LISTING_STATUS DEFAULT 'pending',
    IS_FEATURED DEFAULT FALSE,
    VIEWS INTEGER DEFAULT 0,
    EXPIRATION_DATE TIMESTAMP,
    CREATEDAT TIMESTAMP
)

LISTING_IMAGES(
    LISTIMAGE_ID PK,
    LISTING_ID FK -> LISTINGS(LISTING_ID),
    IMAGE_URL VARCHAR(255) NOT NULL,
    IS_PRIMARY DEFAULT FALSE,
    DISPLAY_ORDER INTEGER,
    UPLOADEDAT TIMESTAMP
)

CHATS(
    CHAT_ID PK,
    USERID FK -> USERS(USERID),
    USERID_AS_SELLER FK -> USERS(USERID),
    LISTING_ID FK -> LISTINGS(LISTING_ID),
    LAST_MESSAGE_AT TIMESTAMP,
    IS_ACTIVE DEFAULT TRUE,
    CREATEDAT TIMESTAMP,
    UNIQUE(USERID, USERID_AS_SELLER, LISTING_ID)
)

MESSAGES(
    MESSAGE_ID PK,
    USERID FK -> USERS(USERID),
    CHAT_ID FK -> CHATS(CHAT_ID),
    CONTENT TEXT NOT NULL,
    MESSAGE_TYPE DEFAULT 'text',
    IS_READ DEFAULT FALSE,
    SENTAT TIMESTAMP
)

PAYMENTS(
    PAYMENT_ID PK,
    USERID FK -> USERS(USERID),
    PRICING_ID FK -> PRICING_PLANS(PRICING_ID),
    LISTING_ID FK -> LISTINGS(LISTING_ID),
    PAYMENT_AMOUNT DECIMAL(10,2) NOT NULL,
    PAYMENT_METHOD VARCHAR(20),
    PAYMENT_STATUS DEFAULT 'pending',
    PAYMENT_REF UNIQUE NOT NULL,
    CREATEDAT TIMESTAMP
)

-- Autres tables: PRICING_PLANS, USER_SUBSCRIPTIONS, RATINGS_N_REVIEWS,
-- FAVORITES, NOTIFICATIONS, REPORTS_N_MISCONDUCT, DEALER_APPLICATIONS,
-- DEALER_DOCUMENTS, VERIFICATION_CODES, USER_BADGES, ACTIVITY_LOGS
\end{lstlisting}

\textbf{Index pour optimisation:}
\begin{itemize}
    \item Index sur USERS(USER\_EMAIL) - recherche utilisateur par email
    \item Index sur LISTINGS(LISTING\_STATUS, IS\_FEATURED) - requêtes de filtrage
    \item Index sur LISTINGS(CAT\_ID) - filtrage par catégorie
    \item Index sur MESSAGES(CHAT\_ID, SENTAT) - historique conversations
    \item Index sur NOTIFICATIONS(USERID, IS\_READ) - notifications non lues
\end{itemize}

\subsection{Script SQL}

Le script SQL complet de création de la base de données est généré automatiquement par Django migrations.

\begin{lstlisting}[language=bash, caption=Génération du schéma SQL]
# Générer le script SQL depuis les models Django
python manage.py sqlmigrate users 0001
python manage.py sqlmigrate listings 0001
python manage.py sqlmigrate messaging 0001
# etc.

# Ou exporter le schéma complet
pg_dump -U umuhuza_admin --schema-only umuhuza > schema.sql
\end{lstlisting}

Les contraintes d'intégrité incluent:
\begin{itemize}
    \item \textbf{Contraintes d'unicité:} USER\_EMAIL, PAYMENT\_REF, SLUG
    \item \textbf{Contraintes de clé étrangère:} CASCADE pour dépendances fortes (ex: images liées à annonce), SET NULL pour références optionnelles
    \item \textbf{Contraintes CHECK:} RATING entre 1 et 5, LISTING\_PRICE $>$ 0
    \item \textbf{Contraintes NOT NULL:} Champs obligatoires métier
\end{itemize}

\section{Design des principales interfaces}

\textit{Note: Les designs Figma doivent être créés séparément et les captures d'écran insérées ici.}

\subsection{Design de la page d'authentification}

\begin{figure}[H]
\centering
% TODO: Insérer capture Figma de la page de connexion
\fbox{\parbox{0.8\textwidth}{
\textbf{TODO: Design Figma à créer}\\
\\
Éléments à inclure:
\begin{itemize}
    \item Formulaire de connexion (email + mot de passe)
    \item Bouton "Se connecter"
    \item Lien "Mot de passe oublié?"
    \item Lien "Créer un compte"
    \item Option "Se souvenir de moi"
    \item Séparation visuelle avec ligne OU
    \item Connexion via réseaux sociaux (future feature)
\end{itemize}
}}
\caption{Maquette de la page d'authentification}
\label{fig:design_login}
\end{figure}

\textbf{Charte graphique:}
\begin{itemize}
    \item Couleur principale: Bleu (#0277BD)
    \item Couleur secondaire: Vert (#2E7D32)
    \item Police: Inter ou Roboto
    \item Design responsive mobile-first
\end{itemize}

\subsection{Design de la page d'accueil}

\begin{figure}[H]
\centering
% TODO: Insérer capture Figma de la page d'accueil
\fbox{\parbox{0.8\textwidth}{
\textbf{TODO: Design Figma à créer}\\
\\
Sections:
\begin{itemize}
    \item Header avec logo, barre de recherche, navigation
    \item Hero section avec call-to-action
    \item Annonces featured en carrousel
    \item Catégories en grille
    \item Annonces récentes en grille
    \item Statistiques de la plateforme
    \item Footer avec liens et informations
\end{itemize}
}}
\caption{Maquette de la page d'accueil}
\label{fig:design_home}
\end{figure}

\subsection{Design d'un formulaire de saisie}

\begin{figure}[H]
\centering
% TODO: Insérer capture Figma du formulaire de création d'annonce
\fbox{\parbox{0.8\textwidth}{
\textbf{TODO: Design Figma à créer}\\
\\
Formulaire multi-étapes:
\begin{itemize}
    \item Étape 1: Sélection catégorie
    \item Étape 2: Informations générales (titre, description, prix, localisation)
    \item Étape 3: Upload de photos (drag \& drop)
    \item Étape 4: Prévisualisation et confirmation
    \item Indicateur de progression visuel
    \item Validation en temps réel
\end{itemize}
}}
\caption{Maquette du formulaire de création d'annonce}
\label{fig:design_form}
\end{figure}

\subsection{Design de la page d'affichage enregistrements}

\begin{figure}[H]
\centering
% TODO: Insérer capture Figma de la page de liste d'annonces
\fbox{\parbox{0.8\textwidth}{
\textbf{TODO: Design Figma à créer}\\
\\
Éléments:
\begin{itemize}
    \item Barre de filtres latérale (catégorie, prix, localisation)
    \item Grille d'annonces avec images, titre, prix, localisation
    \item Badges (Featured, Nouveau, Urgent)
    \item Boutons d'action (Favoris, Partager)
    \item Pagination en bas de page
    \item Tri par (Plus récent, Prix croissant/décroissant, Popularité)
    \item Vue alternative en liste
\end{itemize}
}}
\caption{Maquette de la page de recherche/liste d'annonces}
\label{fig:design_listing}
\end{figure}

% FIN DE LA PARTIE 2
% Continuer avec Chapitre 4 dans le document principal
